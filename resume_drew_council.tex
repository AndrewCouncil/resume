\documentclass{resume}

\usepackage[left=0.4 in,top=0.4in,right=0.4 in,bottom=0.4in]{geometry} % Document margins
% \usepackage{inconsolata}
\newcommand{\tab}[1]{\hspace{.2667\textwidth}\rlap{#1}} 
\newcommand{\itab}[1]{\hspace{0em}\rlap{#1}}
\name{Drew Council} % Your name
\email{andrew.p.council@gmail.com} % Your email
\phone{770-833-1759}
\linkedin{https://www.linkedin.com/in/drew-council/} % Your LinkedIn profile
\github{https://github.com/AndrewCouncil/} % Your GitHub URL
% % You can merge both of these into a single line, if you do not have a website.
% \address{+1(123) 456-7890 \\ San Francisco, CA} 
% \address{\href{mailto:contact@faangpath.com}{contact@faangpath.com} \\ \href{https://linkedin.com/company/faangpath}{linkedin.com/company/faangpath} \\ \href{www.faangpath.com}{www.faangpath.com}}  %

\begin{document}
%----------------------------------------------------------------------------------------
%	OBJECTIVE
%----------------------------------------------------------------------------------------

% \begin{rSection}{OBJECTIVE}

% {Software Engineer with 2+ years of experience in XXX, seeking full-time XXX roles.}


% \end{rSection}

%----------------------------------------------------------------------------------------
%	EDUCATION SECTION
%----------------------------------------------------------------------------------------

\begin{rSection}{Education}

\textbf{Bachelor of Computer Science, Electrical \& Computer Engineering}, Duke University \hfill {Graduating 2024}\\
Completed classes in Data Structures/Algorithms, Computer Architecture, and Product Management. \hfill {GPA: 3.7}


\end{rSection}

%----------------------------------------------------------------------------------------
%	WORK EXPERIENCE SECTION
%----------------------------------------------------------------------------------------

\begin{rSection}{Work Experience}

\textbf{Software Engineer} \hfill Jan 2022 - Present\\
\href{https://botbuilt.com}{BotBuilt Robotics} \hfill \textit{Durham, NC}
\begin{itemize}
    \itemsep -3pt {} 
    \item Worked with other engineers to develop software for robotic construction of houses in a full-time capacity.
    \item Introduced unit testing to software stack with coverage reporting to a multi-repository ROS2 workspace.
    \item Designed embedded hardware to communicate to a ROS2 network and provide services for physical control.
    \item Spearheaded a CI for testing and development including automatic pull request Docker container testing.
    \item Containerized ROS2 applications into standardized Docker images for several platforms and architectures.
    \item Integrated AWS container hosting for automated Docker image builds with error reporting.
    \item Worked on an agile startup software team with a high degree of personal project control and responsibility.
    \end{itemize}
 
\textbf{Teaching Assistant} \hfill Aug 2021 - Dec 2021\\
\href{https://fyd.duke.edu/}{Duke First Year Engineering Design} \hfill \textit{Durham, NC}
\begin{itemize}
    \itemsep -3pt {} 
    \item Assisted small groups of students in project management throughout a semester-long design challenge.
    \item Provided technical expertise in actuators, programming embedded systems, sensors, and PCB design.\
\end{itemize}

\end{rSection} 

%----------------------------------------------------------------------------------------
%	LEADERSHIP EXPERIENCE SECTION
%----------------------------------------------------------------------------------------

\begin{rSection}{Leadership Experience}

\textbf{Software Subteam Lead} \hfill Jun 2020 - Jun 2022\\
\href{https://duke-robotics.com/}{Duke University Robotics Club} \hfill \textit{Durham, NC}
\begin{itemize}
    \itemsep -3pt {} 
    \item Competed in annual \href{https://robonation.org/programs/robosub/}{RoboSub robot competition}, where students design fully autonomous submarine robots to complete a variety of tasks underwater.
    \item Coordinated a $25+$ member agile environment team using ROS, Docker, and Git to manage a shared codebase.
    \item Implemented PID, Sensor Fusion, Computer Vision, and SMACH to improve robot accuracy and capability.
    \item Earned 1st in Propulsion System, 3rd in Sensor optimization in 2021; 1st in technical report in 2021 and 2022.    
    \end{itemize}

\end{rSection} 

%----------------------------------------------------------------------------------------
%	PROJECTS SECTION
%----------------------------------------------------------------------------------------

\begin{rSection}{Projects}

\textbf{Aelevate Bike Trainer}, Duke Product Design \hfill{Sep 2022 - Dec 2022}
\begin{itemize}
    \itemsep -3pt {} 
    \item Designed a bike trainer that allows users to simulate riding on a variety of terrains with varied resistance.
    \item Used PlatformIO to write C++ Arduino software to read sensors and control motors.
    \item Created serial interface to communicate with a custom desktop application for user control.
    \end{itemize}

\textbf{Typeracer-style Arcade Game}, Duke Computer Engineering \hfill{Sep 2022 - Dec 2022}
\begin{itemize}
    \itemsep -3pt {} 
    \item Implemented a head-to-head typing arcade game including display and keyboard drivers on an FPGA.
    \item Used Verilog to implement a MIPS-like pipelined processor to process input and generate graphics.
    \end{itemize}

\textbf{Room Availability Detection System}, Duke Engineering \hfill{Sep 2020 - May 2021}
\begin{itemize}
    \itemsep -3pt {} 
    \item Used a Raspberry Pi and Python software to collect PIR sensor data and detect a person's presence.
    \item Communicated this sensor data over a custom web API to publish the availability of a Duke Music piano practice room to allow for any student to check remotely as requested by client.
    \end{itemize}

\end{rSection} 

%----------------------------------------------------------------------------------------
%	SKILLS SECTION
%----------------------------------------------------------------------------------------
\begin{rSection}{Skills}

    Python, C, Linux, ROS, Docker, Git, Bash, Arduino, Verilog, Raspberry Pi, EAGLE, KiCad, AWS, C$++$, Java 

\end{rSection}

\end{document}