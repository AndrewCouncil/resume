\documentclass{resume}

\usepackage[left=0.4 in,top=0.4in,right=0.4 in,bottom=0.4in]{geometry} % Document margins
% \usepackage{inconsolata}
\newcommand{\tab}[1]{\hspace{.2667\textwidth}\rlap{#1}} 
\newcommand{\itab}[1]{\hspace{0em}\rlap{#1}}
\name{Drew Council} % Your name
\email{apc41@duke.edu} % Your email
\phone{770-833-1759}
\linkedin{https://www.linkedin.com/in/drew-council/} % Your LinkedIn profile
\github{https://github.com/AndrewCouncil/} % Your GitHub URL
% % You can merge both of these into a single line, if you do not have a website.
% \address{+1(123) 456-7890 \\ San Francisco, CA} 
% \address{\href{mailto:contact@faangpath.com}{contact@faangpath.com} \\ \href{https://linkedin.com/company/faangpath}{linkedin.com/company/faangpath} \\ \href{www.faangpath.com}{www.faangpath.com}}  %

\begin{document}
%----------------------------------------------------------------------------------------
%	OBJECTIVE
%----------------------------------------------------------------------------------------

% \begin{rSection}{OBJECTIVE}

% {Software Engineer with 2+ years of experience in XXX, seeking full-time XXX roles.}


% \end{rSection}

%----------------------------------------------------------------------------------------
%	EDUCATION SECTION
%----------------------------------------------------------------------------------------

\begin{rSection}{Education}

\textbf{Bachelor of Computer Science, Electrical \& Computer Engineering}, Duke University \hfill {Graduating 2024}\\
Relevant coursework in Data Structures/Algorithms, Operating Systems, and Computer Networking. \hfill {GPA: 3.8}


\end{rSection}

%----------------------------------------------------------------------------------------
%	SKILLS SECTION
%----------------------------------------------------------------------------------------
\begin{rSection}{Skills}
    Linux, Docker, Git, Bash, Python, C, C$++$, GDB, Arduino, Raspberry Pi, AWS, Rust, Debian/Ubuntu, RHEL, Yocto Linux
\end{rSection}

%----------------------------------------------------------------------------------------
%	WORK EXPERIENCE SECTION
%----------------------------------------------------------------------------------------

% \begin{rSection}{Work Experience}
\begin{rSection}{Experience}

\textbf{Software Engineer} \hfill Jan 2022 - Present\\
\href{https://botbuilt.com}{BotBuilt} \hfill \textit{Durham, NC}
\begin{itemize}
    \itemsep -3pt {} 
    \item Collaborated with engineers in an agile team to develop Linux software for robotic construction of houses.
    \item Headed company DevOps, including managing CI/CD, developer tools, and embedded software deployment.
    \item Maintained Linux tooling for developer machines, local servers, and AWS cloud infrastructure.
    \item Utilized Linux kernel configurations to incorporate hardware devices on embedded systems in a low-latency environment.
    \item Implemented secure networked developer access to shared computing and storage resources.
    \item Containerized robotics applications into Docker images for several platforms and hardware configurations.
    \item Spearheaded CI for testing and development, including automatic Docker container testing on pull requests.
    \end{itemize}
 
\textbf{Software Subteam Lead} \hfill Jun 2020 - Aug 2022\\
\href{https://duke-robotics.com/}{Duke University Robotics Club} \hfill \textit{Durham, NC}
\begin{itemize}
    \itemsep -3pt {} 
    \item Competed in annual \href{https://robonation.org/programs/robosub/}{RoboSub robot competition}, where we designed a fully autonomous submarine robot to complete a variety of complex maneuvering and manipulation tasks in an unfamiliar underwater environment.
    \item Earned 1st in Propulsion System, 3rd in Sensor optimization in 2021; 1st in technical report in 2021 and 2022.    
    \item Coordinated a $25+$ member team using Docker and Git to manage a shared Ubuntu-based codebase.
    \item Implemented multiple user graphical passthrough access for development on deployed robot hardware.
\end{itemize}

\textbf{Teaching Assistant} \hfill Aug 2021 - Dec 2021\\
\href{https://fyd.duke.edu/}{Duke First Year Engineering Design} \hfill \textit{Durham, NC}
\begin{itemize}
    \itemsep -3pt {} 
    \item Guided small groups of students in project management throughout a semester-long design challenge.
    \item Provided technical expertise in Rasperry Pi Linux, embedded systems, sensors, and PCB design.\
\end{itemize}

% \end{rSection} 

%----------------------------------------------------------------------------------------
%	LEADERSHIP EXPERIENCE SECTION
%----------------------------------------------------------------------------------------

% \begin{rSection}{Leadership Experience}

\end{rSection} 

%----------------------------------------------------------------------------------------
%	PROJECTS SECTION
%----------------------------------------------------------------------------------------

\begin{rSection}{Projects}

\textbf{Cell Robots Research}, Duke General Robotics Lab \hfill{Feb 2023 - Present}
\begin{itemize}
    \item Designed omnidirectional grounded robots for research in distributed control and communication of robot swarms.
    \item Researched algorithms to network robot swarm nodes in a dynamic and unpredictable environment.
\end{itemize}

\textbf{Custom Routing Information Protocol}, Duke Computer Engineering \hfill{Sep 2022 - Dec 2022}
\begin{itemize}
    \itemsep -3pt {} 
    \item Developed Linux software in C to route packets through complex network topologies containing multiple nodes.
    \item Handled network topology changes with forwarding table updates to correctly route packets to destination addresses.
    \item Utilized sockets to send and receive packets over UDP and TCP.
    \item Used traceroute, ping, and wireshark to debug and test network functionality.
\end{itemize}

\textbf{xv6 UNIX Additions}, Duke Computer Science \hfill{Sep 2022 - Dec 2022}
\begin{itemize}
    \itemsep -3pt {} 
    \item Added various advanced OS features and performance improvements to the xv6 UNIX operating system.
    \item Implemented kernel threads with system calls and context switching to allow for concurrent processes.
    \item Modified memory allocation to lazily allocate memory only on use to improve performance.
    \item Implemented copy-on-write fork memory allocation to reduce fork resource consumption.
    \item Used GDB and valgrind to debug and test operating system functionality.
\end{itemize}

\textbf{Room Availability Detection System}, Duke Engineering First Year Design \hfill{Sep 2020 - May 2021}
\begin{itemize}
    \itemsep -3pt {} 
    \item Used a Raspberry Pi and Python to analyze and error correct PIR sensor data and detect a person's presence.
    \item Communicated this sensor data over a custom web API to publish room status and use statistics to a website.
\end{itemize}

\end{rSection} 

\end{document}