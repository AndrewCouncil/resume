\documentclass{resume}

\usepackage[left=0.4 in,top=0.4in,right=0.4 in,bottom=0.4in]{geometry} % Document margins
% \usepackage{inconsolata}
\newcommand{\tab}[1]{\hspace{.2667\textwidth}\rlap{#1}} 
\newcommand{\itab}[1]{\hspace{0em}\rlap{#1}}
\name{Drew Council} % Your name
\email{andrew.p.council@gmail.com} % Your email
\phone{770-833-1759}
\linkedin{https://www.linkedin.com/in/drew-council/} % Your LinkedIn profile
\github{https://github.com/AndrewCouncil/} % Your GitHub URL
% % You can merge both of these into a single line, if you do not have a website.
% \address{+1(123) 456-7890 \\ San Francisco, CA} 
% \address{\href{mailto:contact@faangpath.com}{contact@faangpath.com} \\ \href{https://linkedin.com/company/faangpath}{linkedin.com/company/faangpath} \\ \href{www.faangpath.com}{www.faangpath.com}}  %

\begin{document}
%----------------------------------------------------------------------------------------
%	OBJECTIVE
%----------------------------------------------------------------------------------------

% \begin{rSection}{OBJECTIVE}

% {Software Engineer with 2+ years of experience in XXX, seeking full-time XXX roles.}


% \end{rSection}

%----------------------------------------------------------------------------------------
%	EDUCATION SECTION
%----------------------------------------------------------------------------------------

\begin{rSection}{Education}

\textbf{Bachelor of Computer Science, Electrical \& Computer Engineering}, Duke University \hfill {Graduating 2024}\\
Completed classes in Data Structures/Algorithms, Computer Architecture, Networking, and Robotics. \hfill {GPA: 3.8}


\end{rSection}

%----------------------------------------------------------------------------------------
%	SKILLS SECTION
%----------------------------------------------------------------------------------------
\begin{rSection}{Skills}
    Python, C, Linux, ROS, ROS2, Docker, Git, Bash, Arduino, Verilog, Raspberry Pi, KiCad, AWS, C$++$, Java, Rust
\end{rSection}

%----------------------------------------------------------------------------------------
%	WORK EXPERIENCE SECTION
%----------------------------------------------------------------------------------------

% \begin{rSection}{Work Experience}
\begin{rSection}{Experience}

\textbf{Software Engineer} \hfill Jan 2022 - Present\\
\href{https://botbuilt.com}{BotBuilt Robotics} \hfill \textit{Durham, NC}
\begin{itemize}
    \itemsep -3pt {} 
    \item Worked with others on an agile team to develop software for robotic construction of houses.
    \item Headed company DevOps, including managing CI/CD, developer tools, and embedded software deployment.
    \item Designed embedded hardware for a ROS2 network, providing services for actuator and sensor control.
    \item Containerized ROS2 applications into Docker images for several platforms and hardware configurations.
    \item Spearheaded CI for testing and development including automatic Docker container testing on pull requests.
    \item Integrated AWS hosting for automated Docker image builds with custom logging and error reporting.
    \end{itemize}
 
\textbf{Software Subteam Lead} \hfill Jun 2020 - Jun 2022\\
\href{https://duke-robotics.com/}{Duke University Robotics Club} \hfill \textit{Durham, NC}
\begin{itemize}
    \itemsep -3pt {} 
    \item Competed in annual \href{https://robonation.org/programs/robosub/}{RoboSub robot competition}, where we designed a fully autonomous submarine robot to complete a variety of complex maneuvering and manipulation tasks in an unfamiliar underwater environment.
    \item Coordinated a $25+$ member agile environment team using ROS, Docker, and Git to manage a shared codebase.
    \item Implemented PID, Sensor Fusion, Computer Vision, and SMACH to improve robot accuracy and capability.
    \item Earned 1st in Propulsion System, 3rd in Sensor optimization in 2021; 1st in technical report in 2021 and 2022.    
\end{itemize}

\textbf{Teaching Assistant} \hfill Aug 2021 - Dec 2021\\
\href{https://fyd.duke.edu/}{Duke First Year Engineering Design} \hfill \textit{Durham, NC}
\begin{itemize}
    \itemsep -3pt {} 
    \item Guided small groups of students in project management throughout a semester-long design challenge.
    \item Provided technical expertise in actuators, programming embedded systems, sensors, and PCB design.\
\end{itemize}

% \end{rSection} 

%----------------------------------------------------------------------------------------
%	LEADERSHIP EXPERIENCE SECTION
%----------------------------------------------------------------------------------------

% \begin{rSection}{Leadership Experience}

\end{rSection} 

%----------------------------------------------------------------------------------------
%	PROJECTS SECTION
%----------------------------------------------------------------------------------------

\begin{rSection}{Projects}

\textbf{Cell Robots Research}, Duke General Robotics Lab \hfill{Feb 2023 - Present}
\begin{itemize}
    \item Designed omnidirectional grounded robot swarm for research in distributed control and communication.
    \item Researched algorithms to control a swarm of robots to perform tasks with low communication and orchestration.
\end{itemize}

\textbf{``Aelevate'' Bike Trainer}, Duke Product Design \hfill{Sep 2022 - Dec 2022}
\begin{itemize}
    \itemsep -3pt {} 
    \item Used PlatformIO to write C++ embedded software to read sensors and control motors on a tilting bike trainer.
    \item Created a serial interface to communicate with a custom GUI application for touchscreen user control.
\end{itemize}

\textbf{FPGA Typeracer-style Arcade Game}, Duke Computer Engineering \hfill{Sep 2022 - Dec 2022}
\begin{itemize}
    \itemsep -3pt {} 
    \item Used Verilog to implement a MIPS-like pipelined processor to process serial input and generate VGA graphics.
    \item Implemented a head-to-head typing arcade game including display and keyboard drivers on an FPGA.
\end{itemize}

\textbf{Room Availability Detection System}, Duke Engineering First Year Design \hfill{Sep 2020 - May 2021}
\begin{itemize}
    \itemsep -3pt {} 
    \item Used a Raspberry Pi and Python to analyze and error correct PIR sensor data and detect a person's presence.
    \item Communicated this sensor data over a custom web API to publish room status and use statistics to a website.
\end{itemize}

\end{rSection} 

\end{document}